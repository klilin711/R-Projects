% Options for packages loaded elsewhere
\PassOptionsToPackage{unicode}{hyperref}
\PassOptionsToPackage{hyphens}{url}
%
\documentclass[
]{article}
\usepackage{amsmath,amssymb}
\usepackage{iftex}
\ifPDFTeX
  \usepackage[T1]{fontenc}
  \usepackage[utf8]{inputenc}
  \usepackage{textcomp} % provide euro and other symbols
\else % if luatex or xetex
  \usepackage{unicode-math} % this also loads fontspec
  \defaultfontfeatures{Scale=MatchLowercase}
  \defaultfontfeatures[\rmfamily]{Ligatures=TeX,Scale=1}
\fi
\usepackage{lmodern}
\ifPDFTeX\else
  % xetex/luatex font selection
\fi
% Use upquote if available, for straight quotes in verbatim environments
\IfFileExists{upquote.sty}{\usepackage{upquote}}{}
\IfFileExists{microtype.sty}{% use microtype if available
  \usepackage[]{microtype}
  \UseMicrotypeSet[protrusion]{basicmath} % disable protrusion for tt fonts
}{}
\makeatletter
\@ifundefined{KOMAClassName}{% if non-KOMA class
  \IfFileExists{parskip.sty}{%
    \usepackage{parskip}
  }{% else
    \setlength{\parindent}{0pt}
    \setlength{\parskip}{6pt plus 2pt minus 1pt}}
}{% if KOMA class
  \KOMAoptions{parskip=half}}
\makeatother
\usepackage{xcolor}
\usepackage[margin=1in]{geometry}
\usepackage{color}
\usepackage{fancyvrb}
\newcommand{\VerbBar}{|}
\newcommand{\VERB}{\Verb[commandchars=\\\{\}]}
\DefineVerbatimEnvironment{Highlighting}{Verbatim}{commandchars=\\\{\}}
% Add ',fontsize=\small' for more characters per line
\usepackage{framed}
\definecolor{shadecolor}{RGB}{248,248,248}
\newenvironment{Shaded}{\begin{snugshade}}{\end{snugshade}}
\newcommand{\AlertTok}[1]{\textcolor[rgb]{0.94,0.16,0.16}{#1}}
\newcommand{\AnnotationTok}[1]{\textcolor[rgb]{0.56,0.35,0.01}{\textbf{\textit{#1}}}}
\newcommand{\AttributeTok}[1]{\textcolor[rgb]{0.13,0.29,0.53}{#1}}
\newcommand{\BaseNTok}[1]{\textcolor[rgb]{0.00,0.00,0.81}{#1}}
\newcommand{\BuiltInTok}[1]{#1}
\newcommand{\CharTok}[1]{\textcolor[rgb]{0.31,0.60,0.02}{#1}}
\newcommand{\CommentTok}[1]{\textcolor[rgb]{0.56,0.35,0.01}{\textit{#1}}}
\newcommand{\CommentVarTok}[1]{\textcolor[rgb]{0.56,0.35,0.01}{\textbf{\textit{#1}}}}
\newcommand{\ConstantTok}[1]{\textcolor[rgb]{0.56,0.35,0.01}{#1}}
\newcommand{\ControlFlowTok}[1]{\textcolor[rgb]{0.13,0.29,0.53}{\textbf{#1}}}
\newcommand{\DataTypeTok}[1]{\textcolor[rgb]{0.13,0.29,0.53}{#1}}
\newcommand{\DecValTok}[1]{\textcolor[rgb]{0.00,0.00,0.81}{#1}}
\newcommand{\DocumentationTok}[1]{\textcolor[rgb]{0.56,0.35,0.01}{\textbf{\textit{#1}}}}
\newcommand{\ErrorTok}[1]{\textcolor[rgb]{0.64,0.00,0.00}{\textbf{#1}}}
\newcommand{\ExtensionTok}[1]{#1}
\newcommand{\FloatTok}[1]{\textcolor[rgb]{0.00,0.00,0.81}{#1}}
\newcommand{\FunctionTok}[1]{\textcolor[rgb]{0.13,0.29,0.53}{\textbf{#1}}}
\newcommand{\ImportTok}[1]{#1}
\newcommand{\InformationTok}[1]{\textcolor[rgb]{0.56,0.35,0.01}{\textbf{\textit{#1}}}}
\newcommand{\KeywordTok}[1]{\textcolor[rgb]{0.13,0.29,0.53}{\textbf{#1}}}
\newcommand{\NormalTok}[1]{#1}
\newcommand{\OperatorTok}[1]{\textcolor[rgb]{0.81,0.36,0.00}{\textbf{#1}}}
\newcommand{\OtherTok}[1]{\textcolor[rgb]{0.56,0.35,0.01}{#1}}
\newcommand{\PreprocessorTok}[1]{\textcolor[rgb]{0.56,0.35,0.01}{\textit{#1}}}
\newcommand{\RegionMarkerTok}[1]{#1}
\newcommand{\SpecialCharTok}[1]{\textcolor[rgb]{0.81,0.36,0.00}{\textbf{#1}}}
\newcommand{\SpecialStringTok}[1]{\textcolor[rgb]{0.31,0.60,0.02}{#1}}
\newcommand{\StringTok}[1]{\textcolor[rgb]{0.31,0.60,0.02}{#1}}
\newcommand{\VariableTok}[1]{\textcolor[rgb]{0.00,0.00,0.00}{#1}}
\newcommand{\VerbatimStringTok}[1]{\textcolor[rgb]{0.31,0.60,0.02}{#1}}
\newcommand{\WarningTok}[1]{\textcolor[rgb]{0.56,0.35,0.01}{\textbf{\textit{#1}}}}
\usepackage{graphicx}
\makeatletter
\def\maxwidth{\ifdim\Gin@nat@width>\linewidth\linewidth\else\Gin@nat@width\fi}
\def\maxheight{\ifdim\Gin@nat@height>\textheight\textheight\else\Gin@nat@height\fi}
\makeatother
% Scale images if necessary, so that they will not overflow the page
% margins by default, and it is still possible to overwrite the defaults
% using explicit options in \includegraphics[width, height, ...]{}
\setkeys{Gin}{width=\maxwidth,height=\maxheight,keepaspectratio}
% Set default figure placement to htbp
\makeatletter
\def\fps@figure{htbp}
\makeatother
\setlength{\emergencystretch}{3em} % prevent overfull lines
\providecommand{\tightlist}{%
  \setlength{\itemsep}{0pt}\setlength{\parskip}{0pt}}
\setcounter{secnumdepth}{-\maxdimen} % remove section numbering
\ifLuaTeX
  \usepackage{selnolig}  % disable illegal ligatures
\fi
\IfFileExists{bookmark.sty}{\usepackage{bookmark}}{\usepackage{hyperref}}
\IfFileExists{xurl.sty}{\usepackage{xurl}}{} % add URL line breaks if available
\urlstyle{same}
\hypersetup{
  pdftitle={RMHI/ARMP Problem Set 1},
  pdfauthor={Koquiun Li Lin 1319881 {[}Word Count: 1142{]}},
  hidelinks,
  pdfcreator={LaTeX via pandoc}}

\title{RMHI/ARMP Problem Set 1}
\author{Koquiun Li Lin 1319881 {[}Word Count: 1142{]}}
\date{}

\begin{document}
\maketitle

Please put your answers here, following the instructions in the
assignment description. Put your answers and word count tallies in the
locations indicated; if none is indicated that means there is no word
count for that question. Remember to knit as you go, and submit the
knitted version of this on Canvas.

\hypertarget{q1}{%
\subsection{Q1}\label{q1}}

\textbf{Q1a}

\begin{Shaded}
\begin{Highlighting}[]
\CommentTok{\# Put your code here}
\FunctionTok{table}\NormalTok{(d}\SpecialCharTok{$}\NormalTok{level, d}\SpecialCharTok{$}\NormalTok{talent)}
\end{Highlighting}
\end{Shaded}

\begin{verbatim}
##              
##               comedy dancing instrument magic other singing
##   fun              3       4          3     2     2       8
##   competitive      5       3          3     4     0       5
\end{verbatim}

\textbf{Q1b}

\begin{Shaded}
\begin{Highlighting}[]
\CommentTok{\# Put your code here}
\NormalTok{d}\SpecialCharTok{$}\NormalTok{talent }\OtherTok{\textless{}{-}} \FunctionTok{factor}\NormalTok{(d}\SpecialCharTok{$}\NormalTok{talent, }\AttributeTok{levels=}\FunctionTok{c}\NormalTok{(}\StringTok{"singing"}\NormalTok{,}\StringTok{"dancing"}\NormalTok{,}\StringTok{"instrument"}\NormalTok{,}\StringTok{"comedy"}\NormalTok{,}\StringTok{"magic"}\NormalTok{,}\StringTok{"other"}\NormalTok{))}
\FunctionTok{table}\NormalTok{(d}\SpecialCharTok{$}\NormalTok{talent)}
\end{Highlighting}
\end{Shaded}

\begin{verbatim}
## 
##    singing    dancing instrument     comedy      magic      other 
##         13          7          6          8          6          2
\end{verbatim}

\emph{ANSWER: The most common talent was singing, with 13 performances.}

\textbf{Q1c}

\begin{Shaded}
\begin{Highlighting}[]
\CommentTok{\# Put your code here}
\FunctionTok{colnames}\NormalTok{(d)[}\DecValTok{2}\NormalTok{] }\OtherTok{\textless{}{-}} \StringTok{"species"}
\FunctionTok{head}\NormalTok{(d)}
\end{Highlighting}
\end{Shaded}

\begin{verbatim}
## # A tibble: 6 x 6
##   name   species level       talent  audience judge
##   <fct>  <fct>   <fct>       <fct>      <dbl> <dbl>
## 1 gladly bear    competitive singing        8     2
## 2 gladly bear    fun         dancing        9    NA
## 3 panda  bear    fun         dancing        8    NA
## 4 snowy  bear    fun         singing        7    NA
## 5 bunny  bunny   competitive magic          8     2
## 6 bunny  bunny   fun         comedy        10    NA
\end{verbatim}

\hypertarget{q2}{%
\subsection{Q2}\label{q2}}

\textbf{Q2a}

\begin{Shaded}
\begin{Highlighting}[]
\CommentTok{\# Put your code here}
\NormalTok{d[(d}\SpecialCharTok{$}\NormalTok{judge}\SpecialCharTok{==}\DecValTok{1} \SpecialCharTok{|}\NormalTok{ d}\SpecialCharTok{$}\NormalTok{judge}\SpecialCharTok{==}\DecValTok{2}\NormalTok{) }\SpecialCharTok{\&}\NormalTok{ d}\SpecialCharTok{$}\NormalTok{audience}\SpecialCharTok{\textgreater{}=}\DecValTok{8}\NormalTok{,]}
\end{Highlighting}
\end{Shaded}

\begin{verbatim}
## # A tibble: 26 x 6
##    name        species level       talent  audience judge
##    <fct>       <fct>   <fct>       <fct>      <dbl> <dbl>
##  1 gladly      bear    competitive singing        8     2
##  2 <NA>        <NA>    <NA>        <NA>          NA    NA
##  3 <NA>        <NA>    <NA>        <NA>          NA    NA
##  4 bunny       bunny   competitive magic          8     2
##  5 <NA>        <NA>    <NA>        <NA>          NA    NA
##  6 cuddly paws bunny   competitive dancing        8     1
##  7 <NA>        <NA>    <NA>        <NA>          NA    NA
##  8 flopsy      bunny   competitive singing       10     1
##  9 <NA>        <NA>    <NA>        <NA>          NA    NA
## 10 <NA>        <NA>    <NA>        <NA>          NA    NA
## # i 16 more rows
\end{verbatim}

\textbf{Q2b}

\begin{Shaded}
\begin{Highlighting}[]
\CommentTok{\# Put your code here}
\NormalTok{d }\SpecialCharTok{\%\textgreater{}\%}
  \FunctionTok{filter}\NormalTok{((judge}\SpecialCharTok{==}\DecValTok{1} \SpecialCharTok{|}\NormalTok{ judge}\SpecialCharTok{==}\DecValTok{2}\NormalTok{),}
\NormalTok{         audience}\SpecialCharTok{\textgreater{}=}\DecValTok{8}\NormalTok{)}
\end{Highlighting}
\end{Shaded}

\begin{verbatim}
## # A tibble: 6 x 6
##   name        species level       talent  audience judge
##   <fct>       <fct>   <fct>       <fct>      <dbl> <dbl>
## 1 gladly      bear    competitive singing        8     2
## 2 bunny       bunny   competitive magic          8     2
## 3 cuddly paws bunny   competitive dancing        8     1
## 4 flopsy      bunny   competitive singing       10     1
## 5 lfb         bunny   competitive comedy         8     1
## 6 shadow      bunny   competitive magic          9     1
\end{verbatim}

\textbf{Q2c}

\emph{ANSWER: Put your answer here. {[}Word count: 105{]}}

The difference in the outputs arises from the way missing values are
treated in base R and tidyverse. In base R, during logical operations
involving NA values on the conditions ``judge'' and ``audience'', the
result will also be NA for rows where the operands contain NA. These
rows are not removed, and will appear in the output. On the contrary,
tidyverse functions like filter() excludes rows by default where the
filtering condition evaluates to NA, so if either the condition on
``judge'' or ``audience'' evaluates to NA for a row, that row is
excluded from the filtered output, leading to fewer rows in the output.

\textbf{Q2d}

\begin{Shaded}
\begin{Highlighting}[]
\CommentTok{\# Put your code here}
\NormalTok{remove\_NA }\OtherTok{\textless{}{-}} \SpecialCharTok{!}\FunctionTok{is.na}\NormalTok{(d}\SpecialCharTok{$}\NormalTok{judge) }\SpecialCharTok{\&} \SpecialCharTok{!}\FunctionTok{is.na}\NormalTok{(d}\SpecialCharTok{$}\NormalTok{audience)}
\NormalTok{d[(d}\SpecialCharTok{$}\NormalTok{judge}\SpecialCharTok{==}\DecValTok{1} \SpecialCharTok{|}\NormalTok{ d}\SpecialCharTok{$}\NormalTok{judge}\SpecialCharTok{==}\DecValTok{2}\NormalTok{) }\SpecialCharTok{\&}\NormalTok{ d}\SpecialCharTok{$}\NormalTok{audience}\SpecialCharTok{\textgreater{}=}\DecValTok{8} \SpecialCharTok{\&}\NormalTok{ remove\_NA,]}
\end{Highlighting}
\end{Shaded}

\begin{verbatim}
## # A tibble: 6 x 6
##   name        species level       talent  audience judge
##   <fct>       <fct>   <fct>       <fct>      <dbl> <dbl>
## 1 gladly      bear    competitive singing        8     2
## 2 bunny       bunny   competitive magic          8     2
## 3 cuddly paws bunny   competitive dancing        8     1
## 4 flopsy      bunny   competitive singing       10     1
## 5 lfb         bunny   competitive comedy         8     1
## 6 shadow      bunny   competitive magic          9     1
\end{verbatim}

\hypertarget{q3}{%
\subsection{Q3}\label{q3}}

\textbf{Q3a}

\begin{Shaded}
\begin{Highlighting}[]
\CommentTok{\# Put your code here}
\NormalTok{dshort }\OtherTok{\textless{}{-}}\NormalTok{ d }\SpecialCharTok{\%\textgreater{}\%}
  \FunctionTok{select}\NormalTok{(}\SpecialCharTok{{-}}\FunctionTok{c}\NormalTok{(judge,audience))}
\NormalTok{dshort}
\end{Highlighting}
\end{Shaded}

\begin{verbatim}
## # A tibble: 42 x 4
##    name        species level       talent    
##    <fct>       <fct>   <fct>       <fct>     
##  1 gladly      bear    competitive singing   
##  2 gladly      bear    fun         dancing   
##  3 panda       bear    fun         dancing   
##  4 snowy       bear    fun         singing   
##  5 bunny       bunny   competitive magic     
##  6 bunny       bunny   fun         comedy    
##  7 cuddly paws bunny   competitive dancing   
##  8 cuddly paws bunny   fun         instrument
##  9 flopsy      bunny   competitive singing   
## 10 flopsy      bunny   fun         instrument
## # i 32 more rows
\end{verbatim}

\textbf{Q3b}

\begin{Shaded}
\begin{Highlighting}[]
\CommentTok{\# Put your code here}
\NormalTok{d2 }\OtherTok{\textless{}{-}}\NormalTok{ dshort }\SpecialCharTok{\%\textgreater{}\%}
  \FunctionTok{pivot\_wider}\NormalTok{(}\AttributeTok{names\_from =} \StringTok{"level"}\NormalTok{,}\AttributeTok{values\_from =} \StringTok{"talent"}\NormalTok{)}
\NormalTok{d2}
\end{Highlighting}
\end{Shaded}

\begin{verbatim}
## # A tibble: 23 x 4
##    name        species competitive fun       
##    <fct>       <fct>   <fct>       <fct>     
##  1 gladly      bear    singing     dancing   
##  2 panda       bear    <NA>        dancing   
##  3 snowy       bear    <NA>        singing   
##  4 bunny       bunny   magic       comedy    
##  5 cuddly paws bunny   dancing     instrument
##  6 flopsy      bunny   singing     instrument
##  7 giganticky  bunny   magic       dancing   
##  8 lfb         bunny   comedy      singing   
##  9 pink bunny  bunny   instrument  singing   
## 10 shadow      bunny   magic       other     
## # i 13 more rows
\end{verbatim}

\textbf{Q3c}

\begin{Shaded}
\begin{Highlighting}[]
\CommentTok{\# optional code here}
\NormalTok{d2\_modify }\OtherTok{\textless{}{-}}\NormalTok{ d }\SpecialCharTok{\%\textgreater{}\%}
  \FunctionTok{pivot\_wider}\NormalTok{(}\AttributeTok{names\_from =} \StringTok{"level"}\NormalTok{,}\AttributeTok{values\_from =} \StringTok{"talent"}\NormalTok{)}
\NormalTok{d2\_modify}
\end{Highlighting}
\end{Shaded}

\begin{verbatim}
## # A tibble: 42 x 6
##    name        species audience judge competitive fun       
##    <fct>       <fct>      <dbl> <dbl> <fct>       <fct>     
##  1 gladly      bear           8     2 singing     <NA>      
##  2 gladly      bear           9    NA <NA>        dancing   
##  3 panda       bear           8    NA <NA>        dancing   
##  4 snowy       bear           7    NA <NA>        singing   
##  5 bunny       bunny          8     2 magic       <NA>      
##  6 bunny       bunny         10    NA <NA>        comedy    
##  7 cuddly paws bunny          8     1 dancing     <NA>      
##  8 cuddly paws bunny          9    NA <NA>        instrument
##  9 flopsy      bunny         10     1 singing     <NA>      
## 10 flopsy      bunny         10    NA <NA>        instrument
## # i 32 more rows
\end{verbatim}

\emph{ANSWER: Put your answer here. {[}Word count: 106{]}}

The difference in the outputs when using d and dshort with the
pivot\_wider function is indeed due to the presence of NA values. When
using a dataset like d that includes NA values, pivot\_wider can lead to
more outputs with NA because it treats NA as a valid level of a factor,
creating additional columns for NA values. On the other hand, dshort
does include columns where there are NA values. However, pivot\_wider
will generate NA values in the output if there is a mismatch in
key-value pairs. This results in fewer outputs and a smaller tibble
because there are fewer factor levels to pivot on.

\textbf{Q3d}

\begin{Shaded}
\begin{Highlighting}[]
\CommentTok{\# Put your code here}
\NormalTok{d2 }\SpecialCharTok{\%\textgreater{}\%}
  \FunctionTok{filter}\NormalTok{(competitive }\SpecialCharTok{==}\NormalTok{ fun }\SpecialCharTok{|} \FunctionTok{is.na}\NormalTok{(competitive) }\SpecialCharTok{|} \FunctionTok{is.na}\NormalTok{(fun))}
\end{Highlighting}
\end{Shaded}

\begin{verbatim}
## # A tibble: 6 x 4
##   name     species competitive fun    
##   <fct>    <fct>   <fct>       <fct>  
## 1 panda    bear    <NA>        dancing
## 2 snowy    bear    <NA>        singing
## 3 tweak    cat     singing     singing
## 4 barky    dog     comedy      <NA>   
## 5 quackers duck    comedy      comedy 
## 6 monkey   monkey  <NA>        singing
\end{verbatim}

\emph{ANSWER: The names of the individuals who broke Rule 1 (i.e., that
everybody needs to participate in both fun and competitive levels) are
panda, snowy, barky and monkey. The names of the individuals who broke
Rule 2 (i.e., that everybody must to do different kinds of talent at the
fun and competitive levels) are tweak and quackers.}

\hypertarget{q4}{%
\subsection{Q4}\label{q4}}

\textbf{Q4a}

\begin{Shaded}
\begin{Highlighting}[]
\CommentTok{\# Put your code here}
\NormalTok{d }\OtherTok{\textless{}{-}}\NormalTok{ d }\SpecialCharTok{\%\textgreater{}\%}
  \FunctionTok{arrange}\NormalTok{(name)}
\NormalTok{d}
\end{Highlighting}
\end{Shaded}

\begin{verbatim}
## # A tibble: 42 x 6
##    name        species  level       talent     audience judge
##    <fct>       <fct>    <fct>       <fct>         <dbl> <dbl>
##  1 barky       dog      competitive comedy            7     3
##  2 black       dog      competitive dancing           7     3
##  3 black       dog      fun         comedy            9    NA
##  4 bunny       bunny    competitive magic             8     2
##  5 bunny       bunny    fun         comedy           10    NA
##  6 cuddly paws bunny    competitive dancing           8     1
##  7 cuddly paws bunny    fun         instrument        9    NA
##  8 doggie      dog      competitive comedy           NA     2
##  9 doggie      dog      fun         singing           9    NA
## 10 douglas     hedgehog competitive singing           9     3
## # i 32 more rows
\end{verbatim}

\textbf{Q4b}

\begin{Shaded}
\begin{Highlighting}[]
\CommentTok{\# Put your code here}
\NormalTok{d\_full }\OtherTok{\textless{}{-}} \FunctionTok{full\_join}\NormalTok{(d, dd)}
\NormalTok{d\_full}
\end{Highlighting}
\end{Shaded}

\begin{verbatim}
## # A tibble: 42 x 7
##    name        species  level       talent     audience judge duration
##    <fct>       <fct>    <fct>       <chr>         <dbl> <dbl>    <dbl>
##  1 barky       dog      competitive comedy            7     3     11.6
##  2 black       dog      competitive dancing           7     3      5.3
##  3 black       dog      fun         comedy            9    NA      9.9
##  4 bunny       bunny    competitive magic             8     2      8.9
##  5 bunny       bunny    fun         comedy           10    NA      8.3
##  6 cuddly paws bunny    competitive dancing           8     1      4.7
##  7 cuddly paws bunny    fun         instrument        9    NA      5.5
##  8 doggie      dog      competitive comedy           NA     2      9.2
##  9 doggie      dog      fun         singing           9    NA      4.1
## 10 douglas     hedgehog competitive singing           9     3      4.4
## # i 32 more rows
\end{verbatim}

\textbf{Q4c}

\begin{Shaded}
\begin{Highlighting}[]
\CommentTok{\# This code has been given to you, you just need to run it}
\NormalTok{dc }\OtherTok{\textless{}{-}} \FunctionTok{cbind}\NormalTok{(d,dd)}
\FunctionTok{head}\NormalTok{(dc)}
\end{Highlighting}
\end{Shaded}

\begin{verbatim}
##          name species       level  talent audience judge    name       level
## 1       barky     dog competitive  comedy        7     3   barky competitive
## 2       black     dog competitive dancing        7     3   black         fun
## 3       black     dog         fun  comedy        9    NA   bunny         fun
## 4       bunny   bunny competitive   magic        8     2  doggie competitive
## 5       bunny   bunny         fun  comedy       10    NA     lfb competitive
## 6 cuddly paws   bunny competitive dancing        8     1 paw paw competitive
##   talent duration
## 1 comedy     11.6
## 2 comedy      9.9
## 3 comedy      8.3
## 4 comedy      9.2
## 5 comedy      8.9
## 6 comedy      9.0
\end{verbatim}

\emph{ANSWER: Put your answer here. {[}Word count: 87{]}}

The output from cbind() results in more columns, totaling 10, whereas
full\_join() yields only 7 columns. Additionally, the tibble after using
cbind() contains 3 repeated columns: ``name'', ``level'', and
``talent'', whereas the columns after using full\_join() are all unique.
These differences arise because cbind() simply combines tibbles,
preserving the original row order from the first tibble and then
appending the second tibble. In contrast, full\_join() performs a
relational join operation, combining rows based on matching key columns
and ensuring that values in common columns are aligned correctly.

\hypertarget{q5}{%
\subsection{Q5}\label{q5}}

\textbf{Q5a}

\begin{Shaded}
\begin{Highlighting}[]
\CommentTok{\# Put your code here}
\NormalTok{df }\SpecialCharTok{\%\textgreater{}\%}
  \FunctionTok{mutate}\NormalTok{(}\AttributeTok{durType =} \FunctionTok{case\_when}\NormalTok{(duration}\SpecialCharTok{\textgreater{}}\DecValTok{10} \SpecialCharTok{\textasciitilde{}} \StringTok{"long"}\NormalTok{,}
\NormalTok{                             duration}\SpecialCharTok{\textless{}}\DecValTok{5} \SpecialCharTok{\textasciitilde{}} \StringTok{"short"}\NormalTok{,}
                             \ConstantTok{TRUE} \SpecialCharTok{\textasciitilde{}} \StringTok{"medium"}\NormalTok{))}
\end{Highlighting}
\end{Shaded}

\begin{verbatim}
## # A tibble: 42 x 8
##    name        species  level       talent     audience judge duration durType
##    <chr>       <chr>    <chr>       <chr>         <dbl> <dbl>    <dbl> <chr>  
##  1 barky       dog      competitive comedy            7     3     11.6 long   
##  2 black       dog      competitive dancing           7     3      5.3 medium 
##  3 black       dog      fun         comedy            9    NA      9.9 medium 
##  4 bunny       bunny    competitive magic             8     2      8.9 medium 
##  5 bunny       bunny    fun         comedy           10    NA      8.3 medium 
##  6 cuddly paws bunny    competitive dancing           8     1      4.7 short  
##  7 cuddly paws bunny    fun         instrument        9    NA      5.5 medium 
##  8 doggie      dog      competitive comedy           NA     2      9.2 medium 
##  9 doggie      dog      fun         singing           9    NA      4.1 short  
## 10 douglas     hedgehog competitive singing           9     3      4.4 short  
## # i 32 more rows
\end{verbatim}

\textbf{Q5b}

\begin{Shaded}
\begin{Highlighting}[]
\CommentTok{\# Put your code here}
\NormalTok{ds }\OtherTok{\textless{}{-}}\NormalTok{ df }\SpecialCharTok{\%\textgreater{}\%}
  \FunctionTok{group\_by}\NormalTok{(talent) }\SpecialCharTok{\%\textgreater{}\%}
  \FunctionTok{summarise}\NormalTok{(}\AttributeTok{medAud =} \FunctionTok{round}\NormalTok{(}\FunctionTok{median}\NormalTok{(audience,}\AttributeTok{na.rm =} \ConstantTok{TRUE}\NormalTok{),}\DecValTok{2}\NormalTok{),}
            \AttributeTok{mnAud =} \FunctionTok{round}\NormalTok{(}\FunctionTok{mean}\NormalTok{(audience,}\AttributeTok{na.rm =} \ConstantTok{TRUE}\NormalTok{),}\DecValTok{2}\NormalTok{),}
            \AttributeTok{sdAud =} \FunctionTok{round}\NormalTok{(}\FunctionTok{sd}\NormalTok{(audience,}\AttributeTok{na.rm =} \ConstantTok{TRUE}\NormalTok{),}\DecValTok{2}\NormalTok{),}
            \AttributeTok{n =} \FunctionTok{n}\NormalTok{(),}
            \AttributeTok{sderrAud =} \FunctionTok{round}\NormalTok{(sdAud}\SpecialCharTok{/}\FunctionTok{sqrt}\NormalTok{(}\FunctionTok{length}\NormalTok{(audience)),}\DecValTok{3}\NormalTok{)) }\SpecialCharTok{\%\textgreater{}\%}
  \FunctionTok{ungroup}\NormalTok{()}
\NormalTok{ds}
\end{Highlighting}
\end{Shaded}

\begin{verbatim}
## # A tibble: 6 x 6
##   talent     medAud mnAud sdAud     n sderrAud
##   <chr>       <dbl> <dbl> <dbl> <int>    <dbl>
## 1 comedy        9    8.29  1.8      8    0.636
## 2 dancing       8    8     1.29     7    0.488
## 3 instrument    7    7.5   1.76     6    0.719
## 4 magic         8.5  7.33  2.73     6    1.12 
## 5 other         7.5  7.5   3.54     2    2.50 
## 6 singing       9    8.5   1.31    13    0.363
\end{verbatim}

\textbf{Q5c}

\emph{ANSWER: Put your answer here. {[}Word count: 90{]}}

Based on the mean audience ratings, ``magic'' is the least popular, and
based on the median audience ratings, ``instrument'' is the least
popular.

In the talent show data, despite the presence of other higher ratings,
the mean rating for ``magic'' is heavily influenced by the extreme value
3, which pulls down the central tendency indicated by the mean. On the
other hand, the median rating for ``instrument'' accurately captures
central tendency by finding the middle value in the ordered dataset. By
disregarding extreme values, the median provides a more robust measure
of central tendency.

\hypertarget{q6}{%
\subsection{Q6}\label{q6}}

\textbf{Q6a}

\begin{Shaded}
\begin{Highlighting}[]
\CommentTok{\# Put your code here}
\NormalTok{d6 }\SpecialCharTok{\%\textgreater{}\%}
  \FunctionTok{ggplot}\NormalTok{(}\AttributeTok{mapping =} \FunctionTok{aes}\NormalTok{(}\AttributeTok{x =}\NormalTok{ talent, }\AttributeTok{y =}\NormalTok{ mnAud, }\AttributeTok{fill =}\NormalTok{ talent)) }\SpecialCharTok{+}
  \FunctionTok{geom\_col}\NormalTok{(}\AttributeTok{alpha=}\FloatTok{0.5}\NormalTok{,}\AttributeTok{show.legend=}\ConstantTok{FALSE}\NormalTok{,}\AttributeTok{colour=}\StringTok{"black"}\NormalTok{) }\SpecialCharTok{+} 
  \FunctionTok{geom\_jitter}\NormalTok{(}\AttributeTok{data=}\NormalTok{d\_full,}\AttributeTok{mapping=}\FunctionTok{aes}\NormalTok{(}\AttributeTok{x=}\NormalTok{talent,}\AttributeTok{y=}\NormalTok{audience,}\AttributeTok{color=}\NormalTok{talent),}
              \AttributeTok{alpha=}\FloatTok{0.7}\NormalTok{ ,}\AttributeTok{show.legend=}\ConstantTok{FALSE}\NormalTok{) }\SpecialCharTok{+}
  \FunctionTok{geom\_errorbar}\NormalTok{(}\FunctionTok{aes}\NormalTok{(}\AttributeTok{ymin =}\NormalTok{ mnAud }\SpecialCharTok{{-}}\NormalTok{ sderrAud, }
                    \AttributeTok{ymax =}\NormalTok{ mnAud }\SpecialCharTok{+}\NormalTok{ sderrAud), }\AttributeTok{width=}\FloatTok{0.2}\NormalTok{) }\SpecialCharTok{+}
  \FunctionTok{scale\_fill\_brewer}\NormalTok{(}\AttributeTok{palette=}\StringTok{"Set1"}\NormalTok{) }\SpecialCharTok{+}
  \FunctionTok{scale\_colour\_brewer}\NormalTok{(}\AttributeTok{palette=}\StringTok{"Set1"}\NormalTok{) }\SpecialCharTok{+}
  \FunctionTok{facet\_wrap}\NormalTok{(}\SpecialCharTok{\textasciitilde{}}\NormalTok{level, }\AttributeTok{scales=}\StringTok{"free"}\NormalTok{) }\SpecialCharTok{+}
  \FunctionTok{theme\_bw}\NormalTok{() }\SpecialCharTok{+}
  \FunctionTok{labs}\NormalTok{(}\AttributeTok{title =} \StringTok{"Audience rating for each kind of talent"}\NormalTok{,}
       \AttributeTok{x =} \StringTok{"Talent"}\NormalTok{,}
       \AttributeTok{y =} \StringTok{"Rating (higher=better)"}\NormalTok{) }\SpecialCharTok{+}
  \FunctionTok{theme}\NormalTok{(}\AttributeTok{axis.text.x =} \FunctionTok{element\_text}\NormalTok{(}\AttributeTok{angle =} \DecValTok{45}\NormalTok{, }\AttributeTok{hjust=}\DecValTok{1}\NormalTok{)) }\SpecialCharTok{+}
  \FunctionTok{scale\_y\_continuous}\NormalTok{(}\AttributeTok{breaks =} \FunctionTok{seq}\NormalTok{(}\DecValTok{0}\NormalTok{, }\DecValTok{10}\NormalTok{, }\AttributeTok{by =} \DecValTok{5}\NormalTok{))}
\end{Highlighting}
\end{Shaded}

\includegraphics{1319881-pset1_files/figure-latex/q6a-1.pdf}

\textbf{Q6b}

\emph{ANSWER: Put your answer here. {[}Word count: 119{]}}

The audience ratings for different types of talent vary between the
``competitive'' and ``fun'' categories. In both categories, comedy seems
to be the most appreciated talent, receiving the highest ratings. This
could suggest that regardless of the context, comedy is universally
enjoyed by the audience; Dancing and instrument also receive high
ratings in both categories, indicating that these talents are
well-received in both competitive and fun settings; Magic receives a
lower rating in the fun category but fares better in the competitive
category. This might suggest that the audience enjoys magic more when
it's presented in a competitive, more intense and serious context;
Singing receives moderate ratings in both categories, suggesting that
it's neither exceptionally well-received nor poorly received.

\hypertarget{q7}{%
\subsection{Q7}\label{q7}}

\textbf{Q7a}

\begin{Shaded}
\begin{Highlighting}[]
\CommentTok{\# Put your code here}
\NormalTok{df }\SpecialCharTok{\%\textgreater{}\%}
  \FunctionTok{ggplot}\NormalTok{(}\AttributeTok{mapping =} \FunctionTok{aes}\NormalTok{(}\AttributeTok{x=}\NormalTok{duration,}\AttributeTok{fill=}\NormalTok{level)) }\SpecialCharTok{+}
  \FunctionTok{geom\_density}\NormalTok{(}\AttributeTok{alpha=}\FloatTok{0.6}\NormalTok{,}\AttributeTok{color =} \StringTok{"black"}\NormalTok{) }\SpecialCharTok{+}
  \FunctionTok{theme\_minimal}\NormalTok{() }\SpecialCharTok{+}
  \FunctionTok{labs}\NormalTok{(}\AttributeTok{title =} \StringTok{"Density plot in terms of level\textquotesingle{}s duration"}\NormalTok{,}
     \AttributeTok{x =} \StringTok{"Duration"}\NormalTok{,}
     \AttributeTok{y =} \StringTok{"Density"}\NormalTok{) }\SpecialCharTok{+}
  \FunctionTok{theme}\NormalTok{(}\AttributeTok{text =} \FunctionTok{element\_text}\NormalTok{(}\AttributeTok{size =} \DecValTok{14}\NormalTok{))}
\end{Highlighting}
\end{Shaded}

\includegraphics{1319881-pset1_files/figure-latex/q7a-1.pdf}

\textbf{Q7b}

\emph{ANSWER: (1) Describe one new thing here. (2) Describe other new
thing here. {[}Word count: 32{]}}

\begin{enumerate}
\def\labelenumi{(\arabic{enumi})}
\tightlist
\item
  I changed the plot's overall theme to minimal style by adding
  theme\_minimal() after geom\_density.
\item
  I adjusted the font size to 14 points of text elements by using
  theme(text = element\_text(size = 14)).
\end{enumerate}

\textbf{Q7c}

\emph{ANSWER: Put your answer here. {[}Word count: 129{]}}

The density plot displays the distribution of duration for ``fun'' and
``competitive'' levels. The x-axis shows the duration, and the y-axis
shows the density. The ``fun'' level shows a single peak around 5
minutes, indicating most ``fun'' levels have a duration close to this
value. The ``competitive'' level shows a bimodal distribution with peaks
around 5 and 10 minutes, suggesting two common durations for it. The
higher peak for ``competitive'' levels is 5 minutes, similar to the peak
for ``fun'' levels, implying a popular shorter duration for both level
types. However, the second peak for ``competitive'' levels around 10
minutes indicates some competitive levels have a longer duration. The
plot also shows a small density for ``fun'' levels beyond 15 minutes,
suggesting a few outliers with extremely long durations.

\hypertarget{q8}{%
\subsection{Q8}\label{q8}}

\emph{ANSWER: Put your answer here. {[}Word count: 71{]}}

Gladly's interpretation of the p-value is incorrect. A p-value is not a
definitive proof of truth of the null hypothesis. Instead, it represents
the probability of observing the test statistic if the null hypothesis
is true. Moreover, changing the alpha threshold after obtaining the
data, known as ``p-hacking'', is problematic. It increases the Type I
error rate and undermines the integrity of the statistical test, which
can lead to incorrect conclusions.

\hypertarget{q9}{%
\subsection{Q9}\label{q9}}

\begin{Shaded}
\begin{Highlighting}[]
\CommentTok{\# get the lowest score}
\NormalTok{lowest }\OtherTok{\textless{}{-}} \FunctionTok{min}\NormalTok{(d}\SpecialCharTok{$}\NormalTok{audience,}\AttributeTok{na.rm=}\ConstantTok{TRUE}\NormalTok{)}
\NormalTok{lowest}
\end{Highlighting}
\end{Shaded}

\begin{verbatim}
## [1] 3
\end{verbatim}

\begin{Shaded}
\begin{Highlighting}[]
\CommentTok{\# get the highest score}
\NormalTok{highest }\OtherTok{\textless{}{-}} \FunctionTok{max}\NormalTok{(d}\SpecialCharTok{$}\NormalTok{audience,}\AttributeTok{na.rm=}\ConstantTok{TRUE}\NormalTok{)}
\NormalTok{highest}
\end{Highlighting}
\end{Shaded}

\begin{verbatim}
## [1] 10
\end{verbatim}

\textbf{Q9a}

\begin{Shaded}
\begin{Highlighting}[]
\CommentTok{\# Put your code here}
\NormalTok{n }\OtherTok{\textless{}{-}} \DecValTok{10}
\NormalTok{p }\OtherTok{\textless{}{-}} \FloatTok{0.7}

\NormalTok{probLowest }\OtherTok{\textless{}{-}} \FunctionTok{dbinom}\NormalTok{(}\AttributeTok{x=}\NormalTok{lowest, }\AttributeTok{size=}\NormalTok{n, }\AttributeTok{prob=}\NormalTok{p)}
\NormalTok{probHighest }\OtherTok{\textless{}{-}} \FunctionTok{dbinom}\NormalTok{(}\AttributeTok{x=}\NormalTok{highest, }\AttributeTok{size=}\NormalTok{n, }\AttributeTok{prob=}\NormalTok{p)}

\NormalTok{probLowest }\OtherTok{\textless{}{-}} \FunctionTok{round}\NormalTok{(probLowest }\SpecialCharTok{*} \DecValTok{100}\NormalTok{, }\DecValTok{1}\NormalTok{)}
\NormalTok{probLowest}
\end{Highlighting}
\end{Shaded}

\begin{verbatim}
## [1] 0.9
\end{verbatim}

\begin{Shaded}
\begin{Highlighting}[]
\NormalTok{probHighest }\OtherTok{\textless{}{-}} \FunctionTok{round}\NormalTok{(probHighest }\SpecialCharTok{*} \DecValTok{100}\NormalTok{, }\DecValTok{1}\NormalTok{)}
\NormalTok{probHighest}
\end{Highlighting}
\end{Shaded}

\begin{verbatim}
## [1] 2.8
\end{verbatim}

\emph{ANSWER: The probability of the lowest score is 0.9\% and the
probability of the highest score is 2.8\%.}

\textbf{Q9b}

\begin{Shaded}
\begin{Highlighting}[]
\CommentTok{\# this code is given}
\NormalTok{df }\OtherTok{\textless{}{-}}\NormalTok{ df }\SpecialCharTok{\%\textgreater{}\%}
  \FunctionTok{mutate}\NormalTok{(}\AttributeTok{prob =} \FunctionTok{pnorm}\NormalTok{(duration,}\AttributeTok{mean=}\FloatTok{6.5}\NormalTok{,}\AttributeTok{sd=}\DecValTok{3}\NormalTok{))}

\CommentTok{\# you may add additional code here if it\textquotesingle{}s useful to answer the question}
\NormalTok{df }\SpecialCharTok{\%\textgreater{}\%}
  \FunctionTok{ggplot}\NormalTok{(}\AttributeTok{mapping =} \FunctionTok{aes}\NormalTok{(}\AttributeTok{x=}\NormalTok{duration)) }\SpecialCharTok{+}
  \FunctionTok{geom\_histogram}\NormalTok{(}\AttributeTok{alpha=}\FloatTok{0.6}\NormalTok{,}\AttributeTok{color =} \StringTok{"black"}\NormalTok{, }\AttributeTok{binwidth =} \FloatTok{0.5}\NormalTok{) }\SpecialCharTok{+}
  \FunctionTok{theme\_bw}\NormalTok{()}
\end{Highlighting}
\end{Shaded}

\includegraphics{1319881-pset1_files/figure-latex/q9b-1.pdf}

\emph{ANSWER: Put your answer here. {[}Word count: 93{]}}

The variable ``prob'' represents the probability of observing a
``duration'' value under the assumption that the true average duration
is 6.5 minutes with a standard deviation of 3 minutes, while p-value is
the probability of finding an observed test statistic, under the
assumption that the null hypothesis is true. From the idea, we can get
that p-value = 1 - prob, indicating that when prob is larger, p-value is
smaller, vice versa. From the plot, we can identify that there is a data
point significant different from previous averages, which is greater
than 15.

\textbf{Q9c}

\emph{ANSWER: Put your answer here. {[}Word count: 123{]}}

No, we cannot draw conclusions about the significance of the entire
variable duration based on a single calculation combining only the
individual prob values. The prob values alone do not provide
comprehensive insights into the distribution, variance, or other
statistical characteristics of the duration data. They are just
individual probabilities associated with each data point and do not
reflect the overall behavior of the duration data. Moreover, this
approach ignores the potential influence of other variables in the
dataset on duration. These variables could also have significant
interactions with duration that are not captured by looking at prob
values alone. Therefore, extra information such as the descriptive
statistics and the correlation coefficient could be helpful in analyzing
the relationship between duration and prob.

\hypertarget{q10}{%
\subsection{Q10}\label{q10}}

\textbf{Q10a}

\emph{ANSWER: Put your answer here. {[}Word count: 90{]}}

A sampling distribution represents the probability distribution of a
statistic obtained from a large number of samples drawn from a
population. The true distribution of audience ratings increases linearly
from 0 to 10. This indicates that higher ratings are more probable than
lower ones. Consequently, in a single timeslot consisting of 30
performances, we expect the range of ratings to be skewed towards higher
values. Panel X best reflects this expected distribution with its
increasing curve, aligning with the true distribution of ratings and the
anticipated skew towards higher values.

\textbf{Q10b}

\emph{ANSWER: Put your answer here. {[}Word count: 97{]}}

With a uniform distribution of audience ratings, ranging from 0 to 10
across 30 performances, each range is equally likely. The sampling
distribution of the range is expected to be approximately symmetric and
bell-shaped due to large sample sizes, making panel V the optimal
choice.

If the true distribution changes, both the sampling distribution of the
range and the mean will change accordingly. However, the mean is
influenced by every value and is sensitive to outliers, while the range
is determined by the extreme values and can vary widely if the true
distribution has a large spread.

\hypertarget{q11}{%
\subsection{Q11}\label{q11}}

\emph{ANSWER: Put your answer here. Does not contribute to your word
count limit.}

In Bunnyland, everyone is hungry because they have all performed
multiple talents, which consumed their energy, so they need tons of
food.

\end{document}
